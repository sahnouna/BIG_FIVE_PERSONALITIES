\section{Introduction}

\begin{itemize}

    \item What is the subject of the study? 

 When working with a large data set, it can be useful to represent the entire data set with Numerical methods to compare the mean of males and the mean of females with the big five personality (Introversion/Extraversion Neuro Agree Openness Conscience ) to see whether it is significant or not and our 
       
                                      
    \item What is the purpose of the study (working hypothesis)?
     
     {\normalsize{\bf  our hypothesis  : }}       
 
 {\normalsize{\bf   H0 :   the mean = 0}}       
                          
 {\normalsize{\bf  H1 :  the mean  ≠ 0}}   

    \item What do we already know about the subject (literature
        review)? Use citations: {\it \citet{Gallant:87} shows that...
        Alternative Forms of the Wald test are considered
        \citep{Breusch&Schmidt:88}.} 
    \item What is the innovation of the study?

    \item Provide an overview of your results.


    \item Outline of the paper:\\
        {\it The paper is organized as follows. The next section describes the
        model under investigation. Section \ref{Sec:Data} describes the data set
        and Section \ref{Sec:Results} presents the results. Finally, Section
        \ref{Sec:Conc} concludes.}

    \item The introduction should not be longer than 4 pages.

\end{itemize}
